\subsubsection{Euler's planar graph formula}
$V-E+F=C+1$. $E\leq 3V-6$ (when $V\geq 3$)

\subsubsection{Pick's theorem}
For simple polygon, when points are all integer, we have $A=\text{\#\{lattice points in the interior\}} + \frac{\text{\#\{lattice points on the boundary\}}}{2} - 1$

\subsubsection{SG}
\noindent 有向图游戏是一个经典的博弈游戏--实际上,大部分的公平组合游戏都可以转换为有向图游戏。在一个
有向无环图中,只有一个起点,上面有一个棋子,两个玩家轮流沿着有向边推动棋子,不能走的玩家判负。

\noindent 对于状态 $x$ 和它的所有 $k$ 个后继状态 $y_1, y_2, \ldots, y_k$,定义 $\operatorname{SG}$ 函数:

$$
\operatorname{SG}(x)=\operatorname{mex}\{\text{SG}(y_1), \operatorname{SG}(y_2), \ldots, \operatorname{SG}(y_k)\}
$$

\noindent 而对于由 $n$ 个有向图游戏组成的组合游戏,设它们的起点分别为 $s_1, s_2, \ldots, s_n$,\\
则有定理:\\
\textbf{当且仅当 $\operatorname{SG}(s_1) \oplus \operatorname{SG}(s_2) \oplus \ldots \oplus \operatorname{SG}(s_n) \neq 0$ 时,这个游戏是先手必胜的。\\
同时,这是这一个组合游戏的游戏状态 $x$ 的 SG 值。}

\noindent 这一定理被称作 \textbf{Sprague-Grundy 定理} (Sprague-Grundy Theorem)。